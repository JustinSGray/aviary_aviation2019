%!TEX encoding = UTF-8 Unicode
% Author: Laurent Dutriaux
\documentclass[a4paper,11pt]{article}
\usepackage[utf8]{inputenc}
\usepackage{fourier} % Utilisation des polices texte
\usepackage{tikz}
%%%<
\usepackage{verbatim}
\usepackage[active,tightpage]{preview}
\PreviewEnvironment{tikzpicture}
\setlength\PreviewBorder{5pt}%
%%%>

\begin{comment}
:Title: Timetable

A simple example of a timetable. The main purpose is to provide an easy input of sequences, namely::

    \node[event] at (Day,beginning at Hour) {Text}+ 
    
with consistent node styles and positioning. 
\end{comment}

\usetikzlibrary[positioning]
\usetikzlibrary{patterns}
\usepackage[french]{babel} % styles français
\title{A simple Timetable}
\author{Laurent Dutriaux}
\date{\today}
\newcommand{\discwidth}{2.9 cm}
\begin{document}

\maketitle

\begin{tikzpicture}[node distance=0 cm,outer sep = 0pt]

\tikzstyle{framework}=[draw, rectangle, minimum height=1cm, minimum width=5*\discwidth, fill=gray!80,anchor=south west]
\tikzstyle{library}=[draw, rectangle, minimum height=1cm, minimum width=\discwidth, fill=gray!60,anchor=south west]
\tikzstyle{analysis}=[draw, rectangle, minimum height=1cm, minimum width=\discwidth, fill=gray!40,anchor=south west]
\tikzstyle{multidisc}=[draw, rectangle, minimum height=1cm, minimum width=5*\discwidth, fill=gray!20,anchor=south west]
\tikzstyle{label}=[minimum height=1cm, minimum width=\discwidth, anchor=south west]

\node[framework] (OpenMDAO) at (0,0) {OpenMDAO};

\node[library] (OpenBEMT) at (0,1) {OpenBEMT};
\node[analysis] (Propeller) at (0,2) {Propeller};

\node[library] (ZapPy) [right = of OpenBEMT] {ZapPy};
\node[analysis] (Electrical) [right = of Propeller] {Electrical};

\node[library] (pyCycle) [right = of ZapPy] {pyCycle};
\node[analysis] (Turboshaft) [right = of Electrical] {Turboshaft};

\node[library] (OpenAeroStruct) [right = of pyCycle] {OpenAeroStruct};
\node[analysis] (Wing) [right = of Turboshaft] {Wing};

\node[library] (Dymos) [right = of OpenAeroStruct] {Dymos};
\node[analysis] (Trajectory) [right = of Wing] {Trajectory};

\node[multidisc] (Multidisciplinary) at (0,3) {Multidisciplinary Model};

\node[label] (Framework) [right = of OpenMDAO] {Framework};
\node[label] (Library) [right = of Dymos] {Library};
\node[label] (Analysis) [right = of Trajectory] {Analysis};
\node[label] (Model) [right = of Multidisciplinary] {Model};

\draw (14.8,0) -- (17.1,0);
\draw (14.8,1) -- (17.1,1);
\draw (14.8,2) -- (17.1,2);
\draw (14.8,3) -- (17.1,3);
\draw (14.8,4) -- (17.1,4);
\end{tikzpicture}


\end{document} 